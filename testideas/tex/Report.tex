
\PassOptionsToPackage{full}{textcomp}
\documentclass[]{tufte-handout}
%\usepackage{fontspec}
%\usepackage{ETbb}


% load babel package and options here
%\usepackage[p,osf]{ETbb} % osf in text, tabular lining figures in math
\usepackage{ETbb} % osf in text, tabular lining figures in math
\usepackage[scaled=.95,type1]{cabin} % sans serif in style of Gill Sans
\usepackage[varqu,varl]{zi4}% inconsolata typewriter
\usepackage[T1]{fontenc} % LY1 also works
\usepackage[libertine,vvarbb]{newtxmath}
%\usepackage[cal=boondoxo,bb=boondox,frak=boondox]{mathalfa}

%\geometry{showframe} % display margins for debugging page layout

\usepackage{graphicx} % allow embedded images
%  \setkeys{Gin}{width=\linewidth,totalheight=\textheight,keepaspectratio}
  \graphicspath{{graphics/}} % set of paths to search for images
\usepackage{amsmath}  % extended mathematics
\usepackage{booktabs} % book-quality tables
%\usepackage{units}    % non-stacked fractions and better unit spacing
\usepackage{multicol} % multiple column layout facilities
\usepackage{multirow} % multiple column layout facilities
\usepackage{lipsum}   % filler text
\usepackage{fancyvrb} % extended verbatim environments
  \fvset{fontsize=\normalsize}% default font size for fancy-verbatim environments
\usepackage{gensymb} % provides symbols like \degree
\usepackage{ragged2e} % enables hyphenation in ragged-right justification
\usepackage[normalize-symbols]{textalpha} %enables \textalpha for alpha symbol etc.

\usepackage{hyperref} % enables styling of href and url
\hypersetup{
    pdftitle={Tutorial X},
    pdfauthor={Barry Linkletter},
    colorlinks=true,
    linkcolor=blue,
    filecolor=magenta,      
    urlcolor=blue,
    pdfborder={0 0 0},
    frenchlinks=false,
    pdfpagemode=FullScreen,
    }

\usepackage{enumitem} % allows resuming enumerate lists.
\usepackage{mathtools}
\usepackage{mhchem}

\usepackage{siunitx} % provides "S" column class for aligning decimals.  


\usepackage{nicefrac}

\usepackage{varioref}

\usepackage{babel}
\usepackage{float}
\usepackage{stackrel}


\usepackage[shortconst]{physconst}

\usepackage[normalem]{ulem}  % provides strikethrough \sout{}

\usepackage{newfloat}
\DeclareFloatingEnvironment[
  fileext = los ,
  listname = {List of Schemes} ,
  name = Scheme
]{scheme}                    % provides scheme numbering for chemical schemes



\newcommand{\Km}{$\rm K_M$}
\newcommand{\Vmax}{$\rm V_{max}$}
\newcommand{\kcat}{$\rm k_{cat}$}



\title{Tutorial \#X: Preminary Report}
\author[Barry Linkletter]{Barry Linkletter}
\date{} % without \date command, current date is supplied


\begin{document}
\justifying


\maketitle% this prints the handout title, author, and date

\begin{abstract}
\noindent This is the final report produced by your unfortunate predecessor. It describes the enzyme assays for wild-type \emph{\textbeta -galactosidase} with PNP-\textbeta-D-Gal. Please use this report to develop your own method for analyzing the remaining data that still needs to be analyzed. Then you may have cake.
\marginnote[-20mm]{This document was produced using the \LaTeX\ typesetting language with the Tufte-handout document class. Images of proteins were created using \textit{UCSF Chimera}. Chemical diagrams were made with \textit{ChemDoodle} and further edited with \textit{Affinity Designer}.}

\end{abstract}





\section{Part 1: Introduction}

This report describes the data analysis for the enzyme assay used to evaluate mutated \emph{\textbeta -galactosidase} enzymes. First we must determine and enzyme concentration that will give reaction rates that are fast enough to be easily followed but no too fast where we cannot collect enough data to determine an initial rate. 

\subsection{The Plate Plan}

We will evaluate three different concentrations of enzyme

\begin{marginfigure}[5mm]

  \caption[0mm]{The natural products daidzin and daidzein compared to our synthetic target, 7-(\textbeta -D-Galactopyranosyloxy)-4'-hydroxyisoflavone} 
  \vspace{2mm}
    \centering
  \includegraphics[scale=0.6]{Daidzin2.pdf}
  \vspace{5mm}
  \label{fig:fig1}
\end{marginfigure}



\nobibliography{}

\end{document}